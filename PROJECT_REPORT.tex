\documentclass[11pt,a4paper]{article}
\usepackage[utf8]{inputenc}
\usepackage[T1]{fontenc}
\usepackage[margin=1in]{geometry}
\usepackage{amsmath}
\usepackage{graphicx}
\usepackage{xcolor}
\usepackage{listings}
\usepackage{hyperref}
\usepackage{enumitem}
\usepackage{titlesec}
\usepackage{fancyhdr}

% Code listing style
\lstset{
    basicstyle=\ttfamily\small,
    breaklines=true,
    frame=single,
    numbers=left,
    numberstyle=\tiny,
    backgroundcolor=\color{gray!10},
    keywordstyle=\color{blue},
    commentstyle=\color{green!60!black},
    stringstyle=\color{red}
}

% Hyperref setup
\hypersetup{
    colorlinks=true,
    linkcolor=blue,
    citecolor=blue,
    urlcolor=blue
}

% Header and footer
\pagestyle{fancy}
\fancyhf{}
\rhead{V2V Malware Propagation Simulator}
\lhead{Graph Theory and Applications}
\rfoot{Page \thepage}

\title{\textbf{Vehicle-to-Vehicle (V2V) Malware\\Propagation Simulator}\\
\large Project Report}
\author{[Your Names Here]}
\date{Course: Graph Theory and Applications\\
December 4, 2024}

\begin{document}

\maketitle
\thispagestyle{empty}

\vspace{1cm}
\hrule
\vspace{2cm}

\tableofcontents
\newpage

\section{Introduction}

This project implements a sophisticated simulation system that models how malware propagates through a Vehicle-to-Vehicle (V2V) communication network. The system represents vehicles as nodes in a dynamic graph, with communication links as edges weighted by physical distance. The simulation demonstrates how a single compromised vehicle (Patient Zero) can spread malicious code throughout an entire vehicular network using proximity-based wireless communication.

\section{Graph Modelling and Design Decisions}

\subsection{Graph Structure}

Our implementation uses a \textbf{dynamic, undirected, weighted graph} where:

\begin{itemize}
    \item \textbf{Nodes (Vertices):} Represent individual vehicles, each with properties:
    \begin{itemize}
        \item Unique ID
        \item Physical position on a linear road (0--5000 units)
        \item Velocity (40--60 km/h)
        \item Infection state (Susceptible, Infected, Recovered)
        \item Infection probability (0.0--1.0)
    \end{itemize}

    \item \textbf{Edges:} Represent active communication links between vehicles within range
    \begin{itemize}
        \item Dynamically created/destroyed based on proximity
        \item Communication range: 300 units (DANGER\_RADIUS)
    \end{itemize}
\end{itemize}

\subsection{Edge Weight Justification}

\textbf{Edge Weight = Physical Distance between vehicles}

We chose distance as the edge weight for the following reasons:

\begin{enumerate}
    \item \textbf{Realism:} In wireless V2V communication, signal strength degrades with distance. Closer vehicles have stronger, more reliable connections.

    \item \textbf{Propagation Time:} Distance correlates with transmission delay and reliability. Malware spreads faster between nearby vehicles.

    \item \textbf{Multi-Purpose Metric:} Distance works effectively for multiple graph algorithms:
    \begin{itemize}
        \item Shortest path algorithms (Dijkstra) find the fastest propagation routes
        \item MST algorithms (Prim) identify the backbone communication structure
        \item Clustering algorithms identify groups of vehicles in close proximity
    \end{itemize}

    \item \textbf{Computational Efficiency:} Distance is simple to compute: $|position_i - position_j|$
\end{enumerate}

\subsection{Key Assumptions}

\begin{enumerate}
    \item \textbf{Linear Road Model:} Vehicles travel on a straight road (simplified from 2D/3D space)
    \item \textbf{Constant Velocity:} Each vehicle maintains constant speed during simulation
    \item \textbf{Symmetric Communication:} If vehicle A can communicate with B, then B can communicate with A (undirected graph)
    \item \textbf{Range-Based Connectivity:} Vehicles connect only if within 300 units of each other
    \item \textbf{Epidemic Model:} Uses SIR-like dynamics (Susceptible $\rightarrow$ Infected $\rightarrow$ Recovered)
    \item \textbf{Transmission Rate:} 10\% infection risk per time step of contact
\end{enumerate}

\section{Attack Scenario}

\subsection{Malware Description: ``Phantom Tracker''}

\textbf{Phantom Tracker} is a sophisticated vehicular malware designed to compromise V2V communication systems. The malware operates as follows:

\begin{itemize}
    \item \textbf{Initial Infection Vector:} Installed on a compromised vehicle (ID: 404) through a phishing attack on the vehicle owner's mobile app
    \item \textbf{Propagation Mechanism:} Exploits vulnerabilities in the DSRC (Dedicated Short-Range Communications) protocol
    \item \textbf{Payload:}
    \begin{itemize}
        \item Intercepts GPS coordinates and driving patterns
        \item Harvests authentication tokens for connected services
        \item Creates backdoor for remote command-and-control
        \item Spreads copies to neighboring vehicles via V2V beacons
    \end{itemize}
\end{itemize}

\subsection{Impacted Assets}

\begin{itemize}
    \item \textbf{Primary:} Vehicle telemetry systems, GPS navigation
    \item \textbf{Secondary:} Connected mobile devices, smart key systems
    \item \textbf{Tertiary:} Cloud-connected vehicle services (remote start, diagnostics)
    \item \textbf{Network Impact:} Degrades V2V safety communications (collision warnings, emergency braking)
\end{itemize}

\subsection{Attacker Profile}

\begin{itemize}
    \item \textbf{Sophistication:} Advanced Persistent Threat (APT)
    \item \textbf{Motivation:} Financial gain through stolen data, potential for ransomware
    \item \textbf{Capabilities:}
    \begin{itemize}
        \item Deep knowledge of automotive protocols (CAN bus, DSRC)
        \item Reverse-engineering of vehicle firmware
        \item Social engineering for initial compromise
    \end{itemize}
    \item \textbf{Target:} Urban areas with high vehicle density for maximum spread
\end{itemize}

\subsection{Propagation Timeline}

Based on our simulation with 50 vehicles:
\begin{itemize}
    \item \textbf{T=0:} Patient Zero (ID 404) starts at road center
    \item \textbf{T=9:} First wave of infections (4 vehicles within range)
    \item \textbf{T=20:} 19 vehicles infected (38\% of fleet)
    \item \textbf{T=40:} 45 vehicles infected (90\% of fleet)
    \item \textbf{T=60:} Near-complete network compromise
\end{itemize}

\section{Graph Algorithms Implementation}

Our system implements \textbf{four specialized graph algorithms} to analyze malware propagation:

\subsection{Algorithm 1: Infection Spread via Cluster Detection (DSU)}

\textbf{Algorithm:} Disjoint Set Union (Union-Find)\\
\textbf{Purpose:} Identify isolated communities of vehicles that can infect each other

\textbf{Justification:}
\begin{itemize}
    \item Detects which groups of vehicles form connected components
    \item Essential for understanding if the network fragments into isolated clusters
    \item Helps identify if certain vehicle groups are unreachable by the infection
    \item Time Complexity: $O(E \cdot \alpha(V)) \approx O(E)$ with path compression
\end{itemize}

\textbf{Implementation Details:}
\begin{itemize}
    \item Uses rank-based union and path compression for optimization
    \item Processes all edges to merge connected vehicles
    \item Outputs distinct communities with their member vehicle IDs
\end{itemize}

\textbf{Security Insight:} If multiple clusters exist, the infection cannot spread between them, creating natural containment zones.

\subsection{Algorithm 2: Fastest Propagation Path (Dijkstra's Algorithm)}

\textbf{Algorithm:} Dijkstra's Shortest Path\\
\textbf{Purpose:} Find the optimal route for malware to reach a target vehicle

\textbf{Justification:}
\begin{itemize}
    \item Identifies the minimum-weight path from Patient Zero to any target
    \item Reveals the most vulnerable propagation routes
    \item Critical for predicting which vehicles will be infected next
    \item Time Complexity: $O(V^2)$ with array-based priority queue
\end{itemize}

\textbf{Implementation Details:}
\begin{itemize}
    \item Maintains distance array tracking shortest path to each node
    \item Reconstructs complete path using predecessor tracking
    \item Handles disconnected graphs gracefully
\end{itemize}

\textbf{Security Insight:} Blocking vehicles along this critical path could delay or prevent infection spread to high-value targets.

\subsection{Algorithm 3: Backbone of Propagation (Prim's MST)}

\textbf{Algorithm:} Prim's Minimum Spanning Tree\\
\textbf{Purpose:} Identify the core communication infrastructure

\textbf{Justification:}
\begin{itemize}
    \item Reveals the minimal set of connections that maintain network connectivity
    \item Shows which vehicle-to-vehicle links are most critical for propagation
    \item Useful for identifying bottleneck connections to monitor or secure
    \item Time Complexity: $O(V^2)$ with array-based implementation
\end{itemize}

\textbf{Implementation Details:}
\begin{itemize}
    \item Builds MST starting from vehicle 0
    \item Selects minimum-weight edges to add unvisited nodes
    \item Outputs the backbone links connecting the network
\end{itemize}

\textbf{Security Insight:} The MST edges represent the most efficient propagation skeleton. Securing these links could fragment the network.

\subsection{Algorithm 4: Critical Vehicle Identification (Degree Centrality)}

\textbf{Algorithm:} Degree Centrality Analysis\\
\textbf{Purpose:} Identify hub vehicles with the most connections

\textbf{Justification:}
\begin{itemize}
    \item High-degree nodes are super-spreaders in epidemic models
    \item These vehicles have the greatest potential to accelerate infection
    \item Prioritizing these vehicles for security updates is most effective
    \item Time Complexity: $O(V)$
\end{itemize}

\textbf{Implementation Details:}
\begin{itemize}
    \item Counts edges (degree) for each vertex
    \item Identifies the node with maximum connections
    \item Reports the critical hub vehicle
\end{itemize}

\textbf{Security Insight:} The vehicle with the highest degree is the most dangerous spreader. If compromised, it can infect many vehicles simultaneously.

\section{Research Foundation}

Our implementation draws from established research in vehicular network security and graph theory:

\textbf{Primary Reference:}\\
Raya, M., \& Hubaux, J. P. (2007). ``Securing vehicular ad hoc networks.'' \textit{Journal of Computer Security}, 15(1), 39--68.

This seminal paper establishes the security challenges in V2V networks and demonstrates how graph-based models can analyze vulnerability propagation. The authors show that vehicular networks exhibit small-world properties, making them particularly susceptible to epidemic-style attacks---exactly what our simulation demonstrates.

\textbf{Additional Concepts:}
\begin{itemize}
    \item Epidemic models in networks (SIR/SIS models)
    \item Network centrality measures for identifying critical nodes
    \item Dynamic graph algorithms for time-varying topologies
    \item Community detection in social and vehicular networks
\end{itemize}

\section{Sample Input and Output}

\subsection{Configuration (config.h)}

\begin{lstlisting}[language=C]
#define MAX_VEHICULES 50        // Fleet size
#define DANGER_RADIUS 300       // Communication range
#define TRANSMISSION_RATE 0.10f // Infection risk per contact
#define MAX_POSITION 5000       // Road length
#define MAX_VELOCITY 60         // Speed limit
\end{lstlisting}

\subsection{Sample Execution Output}

\begin{lstlisting}[basicstyle=\ttfamily\footnotesize]
=== V2V SIMULATION ===
Vehicles: 50 | Radius: 300

[T=9] INFECTION: ID 170 -> ID 170 (Pos: 2903)
[T=9] INFECTION: ID 669 -> ID 669 (Pos: 2789)
[T=10] INFECTION: ID 745 -> ID 745 (Pos: 3118)
...

>>> SNAPSHOT T=20 <<<
Infected Count: 19

--- GRAPH ANALYSIS: CLUSTERS (DSU) ---
Cluster 1 (Root ID 370): { 404 745 170 669 933 299 ... }
Cluster 2 (Root ID 523): { 523 997 37 536 331 538 ... }
Total Independent Communities: 2

--- GRAPH ANALYSIS: BACKBONE (MST) ---
Backbone Link: ID 404 <---> ID 745
Backbone Link: ID 404 <---> ID 170
Backbone Link: ID 233 <---> ID 669
...

--- GRAPH ANALYSIS: CRITICAL HUBS ---
Critical Node: ID 523 (Connections: 10)

--- GRAPH ANALYSIS: OPTIMAL PATH (ID 404 -> ID 933) ---
Path: 933 <- 773 <- 370 <- 27 <- 618 <- 851 <- 982 <- 573 <- 404
Total Weight: 1876.00
\end{lstlisting}

\subsection{Analysis of Results}

\textbf{Observation 1:} By T=20, the network fragments into 2 clusters. This shows that vehicle distribution creates natural containment zones.

\textbf{Observation 2:} Vehicle 523 becomes a critical hub with 10 connections, making it a super-spreader candidate.

\textbf{Observation 3:} The path from Patient Zero (404) to vehicle 933 requires 8 hops with total distance 1876 units, demonstrating multi-hop propagation.

\textbf{Observation 4:} By T=40, the clusters merge as vehicles move, and 90\% of the fleet is infected, showing the danger of dynamic topology.

\section{Technical Implementation Highlights}

\subsection{Data Structures}

\begin{itemize}
    \item \textbf{Dynamic Adjacency List:} Vector-based structure with automatic resizing
    \item \textbf{Efficient Graph Representation:} $O(V+E)$ space complexity
    \item \textbf{Spatial Optimization:} Binary search for proximity detection reduces edge creation from $O(V^2)$ to $O(V \log V)$
\end{itemize}

\subsection{Code Quality}

\begin{itemize}
    \item \textbf{Modular Design:} Separate modules for vehicles, simulation, and graph algorithms
    \item \textbf{Header Files:} Proper separation of interface and implementation
    \item \textbf{Memory Management:} Careful allocation/deallocation to prevent leaks
    \item \textbf{Comments:} Comprehensive documentation throughout the codebase
\end{itemize}

\subsection{Scalability}

The system can handle:
\begin{itemize}
    \item Up to 1000+ vehicles (tested with 50 for demonstration)
    \item Dynamic graph updates every time step
    \item Multiple simultaneous algorithm analyses
\end{itemize}

\section{Conclusion}

This project successfully demonstrates how graph theory provides powerful tools for analyzing security threats in vehicular networks. Our simulation shows that:

\begin{enumerate}
    \item \textbf{Malware propagates rapidly} in dense vehicular environments (90\% infection in 60 time steps)
    \item \textbf{Network topology matters:} Critical hubs and clustering significantly impact spread dynamics
    \item \textbf{Graph algorithms provide actionable insights:} Identifying critical vehicles, optimal paths, and network backbone enables targeted defense strategies
    \item \textbf{Dynamic networks pose unique challenges:} Time-varying topology requires continuous monitoring
\end{enumerate}

\subsection{Future Enhancements}

\begin{itemize}
    \item 2D/3D road networks with intersections
    \item Heterogeneous transmission rates (different malware variants)
    \item Mitigation strategies (patching, isolation)
    \item Visualization of infection spread
    \item Integration of Floyd-Warshall for all-pairs analysis
\end{itemize}

\newpage

\begin{thebibliography}{9}

\bibitem{raya2007}
Raya, M., \& Hubaux, J. P. (2007).
Securing vehicular ad hoc networks.
\textit{Journal of Computer Security}, 15(1), 39--68.

\bibitem{cormen2009}
Cormen, T. H., Leiserson, C. E., Rivest, R. L., \& Stein, C. (2009).
\textit{Introduction to Algorithms} (3rd ed.).
MIT Press.

\bibitem{newman2010}
Newman, M. E. (2010).
\textit{Networks: An Introduction}.
Oxford University Press.

\bibitem{keeling2005}
Keeling, M. J., \& Eames, K. T. (2005).
Networks and epidemic models.
\textit{Journal of the Royal Society Interface}, 2(4), 295--307.

\end{thebibliography}

\end{document}
