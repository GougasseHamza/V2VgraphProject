\documentclass[11pt,a4paper]{article}
\usepackage[utf8]{inputenc}
\usepackage[T1]{fontenc}
\usepackage[margin=1in]{geometry}
\usepackage{amsmath}
\usepackage{graphicx}
\usepackage{xcolor}
\usepackage{listings}
\usepackage{hyperref}
\usepackage{enumitem}
\usepackage{titlesec}
\usepackage{fancyhdr}

% Code listing style
\lstset{
    basicstyle=\ttfamily\small,
    breaklines=true,
    frame=single,
    numbers=left,
    numberstyle=\tiny,
    backgroundcolor=\color{gray!10},
    keywordstyle=\color{blue},
    commentstyle=\color{green!60!black},
    stringstyle=\color{red}
}

% Hyperref setup
\hypersetup{
    colorlinks=true,
    linkcolor=blue,
    citecolor=blue,
    urlcolor=blue
}

% Header and footer
\pagestyle{fancy}
\fancyhf{}
\rhead{V2V Malware Propagation Simulator}
\lhead{Graph Theory and Applications}
\rfoot{Page \thepage}

\title{\textbf{Vehicle-to-Vehicle (V2V) Malware\\Propagation Simulator}\\
\large Project Report}
\author{[Your Names Here]}
\date{Course: Graph Theory and Applications\\
December 4, 2024}

\begin{document}

\maketitle
\thispagestyle{empty}

\vspace{1cm}
\hrule
\vspace{2cm}

\tableofcontents
\newpage

\section{Introduction}

This project simulates malware propagation in Vehicle-to-Vehicle (V2V) networks using graph theory. Vehicles are nodes, communication links are edges weighted by distance, and we analyze how infection spreads from a Patient Zero through proximity-based wireless communication.

\section{Graph Model}

\subsection{Structure}

We use a \textbf{dynamic, undirected, weighted graph}:

\begin{itemize}
    \item \textbf{Nodes:} Vehicles with ID, position (0--5000), velocity (40--60 km/h), and infection state
    \item \textbf{Edges:} Communication links within 300 units, weighted by distance
    \item \textbf{Dynamic:} Edges update as vehicles move
\end{itemize}

\subsection{Edge Weight: Distance}

Distance between vehicles is used because: (1) it models realistic signal degradation, (2) correlates with transmission delay, (3) works for multiple algorithms (Dijkstra, MST, clustering), and (4) is computationally simple: $|position_i - position_j|$.

\subsection{Key Assumptions}

Linear road, constant vehicle velocity, symmetric communication, 300-unit range, SIR epidemic model, 10\% infection rate per contact.

\section{Scenario: "Phantom Tracker" Malware}

\textbf{Description:} Vehicular malware exploiting DSRC protocol vulnerabilities to intercept GPS data, harvest authentication tokens, and create backdoors for remote control.

\textbf{Initial Infection:} Phishing attack on vehicle owner's mobile app (Patient Zero: Vehicle ID 404).

\textbf{Impact:} Compromises telemetry systems, GPS, connected devices, and degrades safety communications.

\textbf{Propagation:} In our 50-vehicle simulation: T=9 (first 4 infections), T=20 (38\% infected), T=40 (90\% infected), T=60 (near-complete compromise).

\section{Graph Algorithms}

\subsection{1. Cluster Detection (Disjoint Set Union)}

\textbf{Purpose:} Identify isolated vehicle communities.

\textbf{Method:} Union-Find with path compression. Time: $O(E \cdot \alpha(V)) \approx O(E)$.

\textbf{Insight:} Multiple clusters create natural containment zones preventing spread.

\subsection{2. Fastest Path (Dijkstra with Min-Heap)}

\textbf{Purpose:} Find optimal propagation route from Patient Zero to target.

\textbf{Method:} Binary min-heap priority queue for efficient vertex selection. Time: $O((V+E) \log V)$.

\textbf{Insight:} Blocking vehicles on critical paths can prevent infection spread.

\subsection{3. Network Backbone (Prim's MST)}

\textbf{Purpose:} Identify core communication infrastructure.

\textbf{Method:} Minimum spanning tree showing minimal connections for connectivity. Time: $O(V^2)$.

\textbf{Insight:} MST edges are the most efficient propagation skeleton to monitor.

\subsection{4. Critical Hubs (Degree Centrality)}

\textbf{Purpose:} Find super-spreader vehicles with most connections.

\textbf{Method:} Count edges per vertex. Time: $O(V)$.

\textbf{Insight:} High-degree nodes accelerate infection and should be prioritized for security.

\section{Research Foundation}

\textbf{Key Reference:} Raya \& Hubaux (2007) - "Securing vehicular ad hoc networks" demonstrates how graph models analyze V2V vulnerability propagation and shows vehicular networks' susceptibility to epidemic attacks.

\textbf{Concepts:} SIR epidemic models, network centrality, dynamic graph algorithms, community detection.

\section{Implementation and Results}

\subsection{Configuration}

\begin{lstlisting}[language=C]
#define MAX_VEHICULES 50        // Fleet size
#define DANGER_RADIUS 300       // Communication range
#define TRANSMISSION_RATE 0.10f // Infection risk
#define MAX_POSITION 5000       // Road length
\end{lstlisting}

\subsection{Output Screenshots}

% INSERT SCREENSHOT: Main simulation header and progress bar
\begin{figure}[h]
\centering
\fbox{\includegraphics[width=0.8\textwidth]{screenshot_simulation_header.png}}
\caption{Simulation progress tracking}
\label{fig:sim_header}
\end{figure}

% INSERT SCREENSHOT: Cluster detection (DSU) results
\begin{figure}[h]
\centering
\fbox{\includegraphics[width=0.8\textwidth]{screenshot_clusters.png}}
\caption{Cluster analysis (DSU)}
\label{fig:clusters}
\end{figure}

% INSERT SCREENSHOT: MST backbone display
\begin{figure}[h]
\centering
\fbox{\includegraphics[width=0.8\textwidth]{screenshot_mst.png}}
\caption{Network backbone (MST)}
\label{fig:mst}
\end{figure}

% INSERT SCREENSHOT: Critical hubs analysis
\begin{figure}[h]
\centering
\fbox{\includegraphics[width=0.8\textwidth]{screenshot_hubs.png}}
\caption{Critical hubs}
\label{fig:hubs}
\end{figure}

% INSERT SCREENSHOT: Dijkstra shortest path result
\begin{figure}[h]
\centering
\fbox{\includegraphics[width=0.8\textwidth]{screenshot_path.png}}
\caption{Optimal path (Dijkstra)}
\label{fig:path}
\end{figure}

\subsection{Key Results}

Network fragments into 2 clusters at T=20, creating containment zones. Vehicle 523 is the critical hub with 10 connections. Optimal path from Patient Zero (404) to vehicle 933: 8 hops, 1876 units. By T=40, 90\% infected as clusters merge.

\section{Technical Details}

\textbf{Data Structures:} Dynamic adjacency list, binary min-heap for Dijkstra ($O(\log V)$ operations), DSU with path compression.

\textbf{Code Organization:} Modular design (vehicles, simulation, algorithms), proper headers, memory management, comprehensive comments.

\textbf{User Interface:} Color-coded terminal output, Unicode graphics, progress bars, interactive menu, visual tree structures, status dashboard.

\textbf{Performance:} $O(V+E)$ space, handles 1000+ vehicles with real-time updates.

\section{Conclusion}

Graph theory effectively analyzes V2V security threats. Key findings: (1) rapid propagation (90\% in 60 steps), (2) topology matters—hubs and clusters control spread, (3) algorithms provide defense insights, (4) min-heap optimization enables real-time analysis.

\textbf{Future Work:} 2D/3D networks, mitigation strategies, Floyd-Warshall integration, graphical visualization, resilience analysis.

\begin{thebibliography}{9}

\bibitem{raya2007}
Raya, M., \& Hubaux, J. P. (2007). Securing vehicular ad hoc networks. \textit{Journal of Computer Security}, 15(1), 39--68.

\bibitem{cormen2009}
Cormen, T. H., et al. (2009). \textit{Introduction to Algorithms} (3rd ed.). MIT Press.

\bibitem{newman2010}
Newman, M. E. (2010). \textit{Networks: An Introduction}. Oxford University Press.

\end{thebibliography}

\end{document}
